\documentclass[a4paper,kul]{kulakarticle} %options: kul or kulak (default)

\usepackage[utf8]{inputenc}
\usepackage[english]{babel}
\usepackage{graphicx}
\usepackage{subcaption}
\newlength{\twosubht}
\newsavebox{\twosubbox}
\graphicspath{{../Figures/}{../Matlab/}{/}}
\usepackage[outdir=./]{epstopdf}

\usepackage{amsmath}
\usepackage{amsthm}
\usepackage{amssymb}
\usepackage{gensymb}
\setcounter{MaxMatrixCols}{21}

\usepackage{etoolbox,refcount}
\usepackage{multicol}

\newcounter{countitems}
\newcounter{nextitemizecount}
\newcommand{\setupcountitems}{%
	\stepcounter{nextitemizecount}%
	\setcounter{countitems}{0}%
	\preto\item{\stepcounter{countitems}}%
}
\makeatletter
\newcommand{\computecountitems}{%
	\edef\@currentlabel{\number\c@countitems}%
	\label{countitems@\number\numexpr\value{nextitemizecount}-1\relax}%
}
\newcommand{\nextitemizecount}{%
	\getrefnumber{countitems@\number\c@nextitemizecount}%
}
\newcommand{\previtemizecount}{%
	\getrefnumber{countitems@\number\numexpr\value{nextitemizecount}-1\relax}%
}
\makeatother    
\newenvironment{AutoMultiColItemize}{%
	\ifnumcomp{\nextitemizecount}{>}{3}{\begin{multicols}{2}}{}%
		\setupcountitems\begin{itemize}}%
		{\end{itemize}%
		\unskip\computecountitems\ifnumcomp{\previtemizecount}{>}{3}{\end{multicols}}{}}

\usepackage{pdflscape}

\date{Academic year 2021 -- 2022}
\address{
  Faculty of Engineering Science \\
  Department of Mechanical Engineering \\
  Control theory \texttt{[H04X3a]}}
\title{Report Assignment 2: Velocity control of the cart}
\author{Matthias Derez, Toon Servaes}


\begin{document}

\maketitle
\section{Introduction}
In this report, two velocity controllers for DC motors are designed, using frequency respons methods. The main criterion states that the velocity controller yieldaangeziens a zero steady-state error on a constant velocity reference. 
\section{Design of the controller}
\subsection{Type of the controller}
To satisfy the criterion of zero steady-state error, multiple controllers can be used. A PI, PID and feedforward controller can all yield a zero steady-state error. The feedforward controller can be especially usefull for tracking. However, as the controller must yield a zero steady-state error on a constant velocity reference and deal with errors caused by disturbances, the feedforward controller will not be used. Since a large bandwidth yields a fast responding system, a high bandwidth seems advantaegous. If the bandwidth is too high though, the high frequency noise has more influence. A trade-off between the two has to be made. Generally the sampling frequency has to be at least 10-20 times larger than bandwidth (\underline{REFERENTIE C8 S82}). Because of this, the PI controller is chosen, as the extra bandwith delivered by the PID part is unnecessary.  
\subsection{Design parameters}
To properly execute the design, some design parameters must be determined. 
\subsection{Limitations on bandwidth}

\section{Validation of the controller}




\end{document}